\documentclass{article}
\usepackage[top=10px]{geometry}
\usepackage{amsmath}
\usepackage[fontsize=18pt]{scrextend}

\author{}
\title{Fizyka - Wzory}
\date{}

\begin{document}
	\maketitle

	\begin{enumerate}
		\item \textbf{Iloczyn wektorowy} $ \\
		 \vec{a} \times \vec{b} = (a_yb_z - a_zb_y)\vec{i} + (a_zb_x - a_xb_z)\vec{j} + (a_xb_y - a_yb_x)\vec{k}
		 $
		\item \textbf{Prędkość wektorowa} $ \\
		\vec{v}_\text{śr} = \frac{\Delta\vec{r}}{\Delta t} = \frac{\vec{r}(t + \Delta t) - \vec{r}(t)}{\Delta t}  [\frac{m}{s}]
		$
		
		\item \textbf{Prędkość średnia} $ \\
		v_\text{śr} = \frac{x_\text{całk}}{t_\text{całk}} = \frac{\sum_{i=1}^{n}x_i}{\sum_{i=1}^{n}t_i}
		$
		
		\item \textbf{Prędkość chwilowa} $ \\
		\vec{v} = \lim\limits_{\Delta t \to 0} \frac{\Delta \vec{r}}{\Delta t} = \frac{d \vec{r}}{dt}
		\\
		v_x = \frac{dx}{dt}
		\\
		v_y = \frac{dy}{dt}
		\\
		v_z = \frac{dz}{dt}
		$
		
		\item \textbf{Przyspieszenie} $ \\
		\vec{a}_\text{śr} = \frac{\Delta \vec{v}}{\Delta t} 
		\\
		\vec{a} = \lim\limits_{\Delta t \to 0} \frac{\Delta \vec{v}}{\Delta t} = \frac{d \vec{v}}{dt}
		\\
		a_x = \frac{dv_x}{dt}
		\\
		a_y = \frac{dv_t}{dt}
		\\
		a_z = \frac{dv_z}{dt} 
		\\
		a_s = \frac{d |\vec{v}|}{dt}
		\\
		a_n = \frac{v^2}{R} \longrightarrow \text{R - promień krzywizny toru}
		\\
		\vec{a} = \vec{a}_s + \vec{a}_n 
		\\
		\vec{a} = \lim\limits_{\Delta t \rightarrow 0} \frac{\Delta \vec{v}}{\Delta t} = \frac{d \vec{v}}{dt} = \frac{d}{dt} (\frac{d \vec{r}}{dt}) = \frac{d^2 \vec{r}}{dt^2} [\frac{m}{s^2}]
		$
		
		\item \textbf{Ruch jednostajny prostoliniowy} $ \\
		\vec{a} = \vec{0} \Rightarrow \vec{v} = \overrightarrow{const} \Rightarrow \vec{r}(t) = \vec{r}_0 + \vec{v} \cdotp t
		\\
		\text{Dla ruchu wzdłuż osi x (opis skalarny):}
		\\
		x(t) = x_0 + v \cdotp t
		$
		
		\item \textbf{Ruch zmienny wzdłuż prostej} $ \\
		\vec{a} = \overrightarrow{const} \Rightarrow \vec{v}(t) = \vec{v}_0 + \vec{a} \cdotp t \Rightarrow \vec{r}(t) = \vec{r}_0 + \vec{v}_0 \cdotp t + \frac{\vec{a} \cdotp t^2}{2}
		\\
		\text{Opis skalarny:}
		\\
		v(t) = \pm v_0 \pm a \cdotp t \Rightarrow x(t) = x_0 \pm v_0 \cdotp t \pm \frac{a \cdotp t^2}{2}
		$
		
		\item \textbf{I zasada dynamiki Newtona} $ \\
		\vec{F}_w = \vec{0} \Leftrightarrow \vec{v} = \overrightarrow{const}
		$
		
		\item \textbf{II zasada dynamiki Newtona} $ \\
		\vec{a} = \frac{\vec{F}_w}{m}
		$
		
		\item \textbf{Pęd} $ \\
		\vec{p} = m \cdotp \vec{v}
		$
		
		\item \textbf{II zasada dynamiki w postaci pędowej} $ \\
		\vec{F} \cdotp \Delta t = \Delta \vec{p} \text{ lub } \vec{F} \cdotp dt = d \vec{p}
		\\
		\vec{F} = \frac{\Delta \vec{p}}{\Delta t} \text{ lub } \vec{F} = \frac{d \vec{p}}{dt}
		\\
		\text{Dla stałej masy:}
		\\
		m = const
		\\
		\vec{F} = \frac{d \vec{p}}{dt} = \frac{d(m \cdotp \vec{v})}{dt} = m \cdotp \frac{d \vec{v}}{dt} = m \cdotp \vec{a}
		$
		
		\item \textbf{Równanie ruchy} $ \\
		\vec{F} = m \cdotp \vec{a}
		$
		
		\item \textbf{III zasada dynamiki Newtona} $ \\
		\vec{F}_{A \to B} = -\vec{F}_{B \to A}
		$
		
	\end{enumerate}
	
\end{document}