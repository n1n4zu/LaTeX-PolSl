\documentclass{article}
\usepackage[top=10px]{geometry}
\usepackage{amsmath,amssymb}
\usepackage[utf8]{inputenc}
\usepackage[T1]{fontenc}
\usepackage[polish]{babel}
\usepackage[fontsize=16pt]{scrextend}

\author{}
\title{Fizyka - Wzory}
\date{}

\begin{document}
	\maketitle
	
	\begin{enumerate}
		
		\item \textbf{Iloczyn wektorowy}:
		\[
		\vec{a} \times \vec{b} = (a_yb_z - a_zb_y)\vec{i} + (a_zb_x - a_xb_z)\vec{j} + (a_xb_y - a_yb_x)\vec{k}
		\]
		
		\item \textbf{Prędkość wektorowa}:
		\[
		\vec{v}_\text{śr} = \frac{\Delta\vec{r}}{\Delta t} = \frac{\vec{r}(t + \Delta t) - \vec{r}(t)}{\Delta t} \quad \left[\frac{\text{m}}{\text{s}}\right]
		\]
		
		\item \textbf{Prędkość średnia}:
		\[
		v_\text{śr} = \frac{x_\text{całk}}{t_\text{całk}} = \frac{\sum \limits_{i=1}^{n}x_i}{\sum \limits_{i=1}^{n}t_i}
		\]
		
		\item \textbf{Prędkość chwilowa}:
		\[
		\vec{v} = \lim_{\Delta t \to 0} \frac{\Delta \vec{r}}{\Delta t} = \frac{d \vec{r}}{dt}
		\]
		\[
		v_x = \frac{dx}{dt}, \quad v_y = \frac{dy}{dt}, \quad v_z = \frac{dz}{dt}
		\]
		
		\item \textbf{Przyspieszenie}:
		\[
		\vec{a}_\text{śr} = \frac{\Delta \vec{v}}{\Delta t}
		\]
		\[
		\quad 
		\vec{a} = \lim_{\Delta t \to 0} \frac{\Delta \vec{v}}{\Delta t} = \frac{d \vec{v}}{dt} = \frac{d^2 \vec{r}}{dt^2}
		\]
		\[
		a_x = \frac{dv_x}{dt}, \quad a_y = \frac{dv_y}{dt}, \quad a_z = \frac{dv_z}{dt}, \quad a_s = \frac{d|\vec{v}|}{dt}, \quad a_n = \frac{v^2}{R}
		\]
		
		\item \textbf{Ruch jednostajny prostoliniowy}:
		\[
		\vec{a} = \vec{0} \implies \vec{v} = \overrightarrow{const} \implies \vec{r}(t) = \vec{r}_0 + \vec{v} \cdot t
		\]
		Dla ruchu wzdłuż osi \(x\):
		\[
		x(t) = x_0 + v \cdot t
		\]
		
		\item \textbf{Ruch zmienny wzdłuż prostej}:
		\[
		\vec{a} = \overrightarrow{const} \implies \vec{v}(t) = \vec{v}_0 + \vec{a} \cdot t
		\]
		\[\quad \vec{r}(t) = \vec{r}_0 + \vec{v}_0 \cdot t + \frac{\vec{a} \cdot t^2}{2}
		\]
		
		\item \textbf{I zasada dynamiki Newtona}:
		\[
		\vec{F}_w = \vec{0} \iff \vec{v} = \overrightarrow{const}
		\]
		
		\item \textbf{II zasada dynamiki Newtona}:
		\[
		\vec{a} = \frac{\vec{F}_w}{m}
		\]
		
		\item \textbf{Pęd}:
		\[
		\vec{p} = m \cdot \vec{v}
		\]
		
		\item \textbf{III zasada dynamiki Newtona}:
		\[
		\vec{F}_{A \to B} = -\vec{F}_{B \to A}
		\]
		
		\item \textbf{Rzut poziomy}:
		\[
		x(t) = x_0 + v_{0x} \cdotp t = v_0 \cdotp t
		\]
		\[
		y(t) = y_0 + v_{0y} \cdotp t - \frac{gt^2}{2} = H - \frac{gt^2}{2}
		\]
		\[
		t = t_c \text{ dla } y = 0
		\]
		\[
		\quad H - \frac{gt^2}{2} = 0
		\]
		\[
		\quad H = \frac{gt^2}{2} \rightarrow t_c = \sqrt{\frac{2H}{g}} \left[ \sqrt{\frac{m}{\frac{m}{s^2}}} = s \right]
		\]
		\[
		x = z \text{ dla } y = 0
		\]
		\[
		\quad z = v_0 \cdotp t = v_0 \cdotp \sqrt{\frac{2H}{g}} \left[ \frac{m}{s} \sqrt{\frac{m}{\frac{m}{s^2}}} = \frac{m}{s} \cdotp s = m \right]
		\]
		\[
		v_x(t) = \frac{dx}{dt} = v_0 \cdotp 1 \cdotp t^{1-1} = v_0 = const
		\]
		\[
		v_y(t) = \frac{dy}{dt} = - \frac{g}{2} 2t = -gt \left[ \frac{m}{s^2}s = \frac{m}{s} \right]
		\]
		
		\item \textbf{Rzut pionowy}:
		\[
		y(t) = 0 + v_0t - \frac{gt^2}{2}
		\]
		\[
		v_y(t) = \frac{dy}{dt} = v_0 - gt
		\]
		\[
		t = t_c \text{ dla } y = 0 \rightarrow v_0t_c = \frac{g}{2} t_c^2
		\]
		
		\item \textbf{Rzut ukośny}:
		\[
		v_{0x} = v_0 \cdotp \cos\alpha
		\]
		\[
		v_{0y} = v_0 \cdotp \sin\alpha
		\]
		\[
		x(t) = x_0 + v_{0x} \cdotp t = v_0 \cdotp t \cdotp \cos\alpha
		\]
		\[
		y(t) = y_0 + v_{0y} \cdotp t - \frac{gt^2}{2} = v_0 \cdotp t \cdotp \sin\alpha - \frac{gt^2}{2}
		\]
		\[
		v_x(t) = v_{0x} = v_0 \cdotp \cos\alpha
		\]
		\[
		v_y(t) = v_{0y} - g \cdotp t = v_0 \cdotp \sin\alpha - g \cdotp t
		\]
		\[
		t = t_c \text{ dla } y = 0
		\]
		\[
		y(t) = v_0 \cdotp t \cdotp \sin\alpha - \frac{g \cdotp t^2}{2}, \quad v_0 \cdotp t_c \cdotp \sin\alpha - \frac{g \cdotp t_c^2}{2} = 0
		\]
		\[
		t_c \left( v_0 \cdotp \sin\alpha - \frac{g \cdotp t_c}{2} \right) = 0 \Rightarrow t_c = 0 \quad \vee \quad t_c = \frac{2v_0 \cdot \sin\alpha}{g}
		\]
		\[
		z = x(t_c) \longrightarrow \text{z - zasięg}
		\]
		\[
		x(t) = v_0 \cdot t \cdotp \cos\alpha \Rightarrow x(t_c) = v_0 \cdotp \frac{2v_0\sin\alpha}{g} \cdotp \cos\alpha = \frac{2v_0^2 \sin\alpha \cos\alpha}{g}
		\]
		\[
		y = H_{max} \text{ dla } t = t_{wzn}
		\]
		\[
		H_{max} = y(t_{wzn}) = v_0 \cdotp t_{wzn} \cdotp \sin\alpha - \frac{g \cdotp t_{wzn}^2}{2} =
		\]
		\[
		 v_0 \cdotp \frac{v_0 \cdotp \sin\alpha}{g} \cdotp \sin\alpha - \frac{g}{2} \cdotp \left( \frac{v_0 \cdotp \sin\alpha}{g} \right)^2 = \frac{v_0^2 \cdotp \sin^2 \alpha}{g} - \frac{v_0^2 \cdotp \sin^2 \alpha}{2 \cdotp g}
		\]
		\[
		H_{max} = \frac{v_0^2 \cdotp \sin^2 \alpha}{2 \cdotp g}
		\]
		
		\item \textbf{II zasada dynamiki w postaci pędowej}:
		\[
		\overrightarrow{F_{zew}} = \frac{d \vec{p}}{dt}
		\]
		\[
		\Downarrow
		\]
		\[
		\overrightarrow{F_{zew}} = \vec{0} \Rightarrow \frac{d \vec{p}}{dt} = 0 \Leftrightarrow \vec{p} = \overrightarrow{const}
		\]
		\[
		\vec{p}_{pocz} = \vec{p}_{\text{końc}}
		\]
		\[
		\vec{p}_{pocz} = \vec{0}, \quad \vec{p}_{\text{końc}}= m_1 \cdotp \overrightarrow{v_1} - m_2 \cdotp \overrightarrow{v_2} = \vec{0}
		\]
		
		\item \textbf{Energia}:
		\text{Energia kinetyczna}
		\[
		E_k = \frac{mv^2}{2}
		\]
		\text{Energia potencjalna}
		\[
		E_p = -G \frac{Mm}{r}
		\]
		\text{W pobliżu pow. Ziemi}
		\[
		E_p = mgh
		\]
		\text{Siła elektrostatyczna (Coulomba)}
		\[
		E_p = \frac{1}{4 \pi \varepsilon_0} \frac{q_1 q_2}{r}
		\]
		\text{Siła sprężystości}
		\[
		E_p = \frac{kx^2}{w}
		\]
		\text{Energia kinetyczna ruchu obrotowego}
		\[
		E_{koi} = \frac{m_i v_i^2}{2} = \frac{m_i r_i^2 \omega^2}{2}
		\]
		\[
		E_{ko} = \sum\limits_{i = 1}^n \frac{m_i r_i^2 \omega^2}{2} = \frac{1}{2} \omega^2 \sum\limits_{i = 1}^n m_i r_i^2
		\]
		\[
		E_{ko} = \frac{I \omega^2}{2}
		\]
		\item \textbf{Praca}:
		\[
		W = \vec{F} \circ \vec{s} = F \cdot s \cdot \cos a
		\]
		\[
		\Delta W_i = F_i \cdot \Delta x
		\]
		\[
		W = \sum\limits_{i = 1}^n F_i \cdot \Delta x
		\]
		\[
		W = \lim\limits_{\Delta x \rightarrow 0} \sum\limits_{i = 1}^n F_i \cdot \Delta x = \int\limits_{x_1}^{x_2} F(x)dx
		\]
		\[
		x = v_0 \cdot t + \frac{a \cdot t^2}{2}
		\]
		\[
		v = v_0 + a \cdot t \Rightarrow a = \frac{v - v_0}{t}
		\]
		\[
		x = \frac{v + v_0}{2} \cdot t
		\]
		\[
		W = F \cdot x = m \cdot a \cdot x = m \left( \frac{v - v_0}{t} \right) \left( \frac{v + v_0}{2} \right)t = \frac{mv^2}{2} - \frac{mv_0^2}{2}
		\]
		\[
		W = E_k - E_{k_0} \longrightarrow \text{Równoważność pracy i energii}
		\]
		\item \textbf{Moc}:
		\[
		P_{śr} = \frac{\Delta W}{\Delta t}
		\]
		\[
		P = \lim\limits_{\Delta t \rightarrow 0} \frac{\Delta W}{\Delta t} = \frac{dW}{dt}
		\]
		\[
		P = \frac{dW}{dt} = \frac{\vec{F} \circ d \vec{s}}{dt} = \vec{F} \circ \vec{v}
		\]
		\item \textbf{Tarcie}:
		\[
		T_k = \mu_k \cdot N
		\]
		\[
		\mu_s \geq \mu_k
		\]
		\[
		T_s = -F
		\]
		\[
		0 < T_s < T_{\text{smax}}
		\]
		\[
		T_{\text{smax}} = \mu_s \cdot N
		\]
		\text{Dla małych prędkości obiektu (wzór Stokesa):}
		\[
		F = 6 \Pi \eta r v \longrightarrow \text{Siła oporu kuli o promieniu } r
		\]
		\[
		\eta \rightarrow \text{lepkość płynu}, v \rightarrow \text{prędkość poruszającego się obiektu}
		\]
		\text{Dla dużych prędkości obiektu (wzór Newtona):}
		\[
		F = C_x \cdot \frac{\rho v^2}{2} \cdot S
		\]
		\[
		\rho \rightarrow \text{gęstość płynu}, S \rightarrow \text{powierzchnia obiektu}
		\]
		\[
		C_x \rightarrow \text{współczynnik oporu płynu zależny od kształtu obiektu}
		\]
		\item \textbf{Siły}:
		\[
		\vec{F}_b = -m \cdot \vec{a}_u
		\]
		\[
		\overrightarrow{F_{\text{bezwł}}} = -m \overrightarrow{a_{\text{ukł odn}}}
		\]
		\[
		\overrightarrow{F_{\text{całk}}} = \overrightarrow{F_{\text{oddz}}} + \overrightarrow{F_{\text{bezwł}}}
		\]
		\[
		a_d = \frac{v^2}{R} = \omega^2 \cdot R
		\]
		\[
		F_{od} = m \cdot a_d
		\]
		\[
		F_{od} = m \cdot a_d \frac{mv^2}{R}
		\]
		\[
		F_{od} = m \cdot a_d = m \cdot \omega^2 \cdot R
		\]
		\[
		\overrightarrow{F_c} = -2m(\vec{\omega} \times \vec{v}) \longrightarrow \text{Siła Coriolisa}
		\]
		\text{Oddziaływanie grawitacyjne:}
		\[
		\text{Wzór Newtowna}
		\]
		\[
		F = G \frac{m \cdot M}{r^2}, G = 6.67 \cdot 10^{-11} \text{N} \cdot \text{m}^2 / \text{kg}^2
		\]
		$
			\text{Wzór uwzględniający informację o zwrocie i kierunku siły w miejscu} 	\newline 
			\text{wskazanym przez wektor } \vec{r}
		$
		\[
		\vec{F} = G \frac{m \cdot M}{r^3} \vec{r}
		\]
		\[
		\text{Siła grawitacji na powierzchni Ziemi}
		\]
		\[
		F = mg, g = 9.81 \text{m} / \text{s}^2 
		\]
		\text{Oddziaływanie elektromagnetyczne}
		\[
		F_c = \frac{1}{4 \pi \varepsilon_0} \cdot \frac{q_1 q_2}{r^2} \longrightarrow \text{prawo Coulomba}
		\]
		\[
		\overrightarrow{F_L} =  q(\vec{v} \times \vec{B}) \longrightarrow \text{siła Lorentza}
		\]
		\text{Siła wyporu}
		\[
		\overrightarrow{F_w} = -\rho V \vec{g}
		\]
		\text{Siła sprężystości}
		\[
		F = E \cdot \frac{S}{L_0} \cdot \Delta L
		\]
		\[
		\vec{F} = -k \cdot \vec{x}
		\]
		\item \textbf{Środek masy układu ciał}:
		\[
		x_s = \frac{\sum\limits_{i = 1}^N m_ix_i}{\sum\limits_{i = 1}^N m_i},
		y_s = \frac{\sum\limits_{i = 1}^N m_iy_i}{\sum\limits_{i = 1}^N m_i},
		z_s = \frac{\sum\limits_{i = 1}^N m_iz_i}{\sum\limits_{i = 1}^N m_i}
		\]
		\[
		\Downarrow
		\]
		\[
		\vec{r}_s = \frac{\sum\limits_{i = 1}^N m_i \vec{r}_i}{\sum\limits_{i = 1}^N m_i}
		\]
		\[
		\vec{r}_s = \frac{\sum\limits_{i = 1}^N m_i \vec{r}_i}{\sum\limits_{i = 1}^N m_i}
		\overset{\Delta m_i \rightarrow 0}{\Rightarrow}
		 \vec{r}_s = \frac{\int rdm}{\int dm} =
		\frac{1}{m} \int rdm
		\]
		\[
		mx_s = \sum\limits_{i = 1}^N m_i x_i
		\]
		\text{Pierwsze różniczkowanie}
		\[
		m \frac{dx_s}{dt} = \sum\limits_{i = 1}^N m_i \frac{dx_i}{dt},
		mv_{sx} = \sum\limits_{i = 1}^N m_i v_{xi}
		\]
		\text{Drugie różniczkowanie}
		\[
		m \frac{dv_{sx}}{dt} = ma_{sx} = \sum\limits_{i = 1}^N m_i a_{xi}
		\]
		\text{Po uwzględnieniu II zasady dynamiki}
		\[
		ma_{sx} = \sum\limits_{i = 1}^N m_i a_{xi} = \sum\limits_{i = 1}^N F_{xi}
		\]
		\[
		\text{Analogicznie robimy dla osi y i z}
		\]
		\[
		\Downarrow
		\]
		\[
		m \overrightarrow{a_s} = \sum\limits_{i = 1}^N \vec{F}_i
		\]
		\[
		m \vec{a}_s = \sum\vec{F}_{zi} \longrightarrow \text{Równanie ruchu}
		\]
		\item \textbf{Ruch obrotowy punktu materialnego}:
		\[
		s = \Delta\phi \cdot r \longrightarrow \text{droga kątowa}
		\]
		\[
		\frac{ds}{dt} = \frac{d \phi}{dt}
		\Rightarrow
		\vec{\omega} = \frac{d \vec{\phi}}{dt} \longrightarrow \text{prędkość kątowa}
		\]
		\[
		\vec{v} = \vec{\omega} \times \vec{r} \longrightarrow \text{prędkość liniowa punktu A}
		\]
		\[
		\vec{\varepsilon} = \frac{d \vec{\omega}}{dt} \longrightarrow \text{przyspieszenie kątowe}
		\]
		\text{Przyspieszenie styczne i dośrodkowe}
		\[
		\vec{a} = \frac{d \vec{v}}{dt} = \frac{d}{dt} (\vec{\omega} \times \vec{r}) =
		\frac{d \vec{\omega}}{dt} \times \vec{r} + \vec{\omega} \times \frac{d \vec{r}}{dt}
		\]
		\[
		\vec{a}_s = \frac{d \vec{\omega}}{dt} \times \vec{r} = \vec{\varepsilon} \times \vec{r} \Rightarrow | \vec{a}_s | = \left| \frac{d \vec{v}}{dt} \right|
		\]
		\[
		\vec{a}_n = \vec{\omega} \times \frac{d \vec{r}}{dt} = \vec{\omega} \times \vec{v} \Rightarrow | \vec{a}_n | = \frac{v^2}{r}
		\]
		\item \textbf{Moment siły}:
		\[
		\vec{M} = \vec{r} \times \vec{F}, \vec{r} \longrightarrow \text{ramię siły}
		\]
		\[
		\vec{r} \text{ II } \vec{F} \Rightarrow \vec{M} = 0		
		\]
		\[
		M = rF\sin\alpha = r_{\perp} \cdot F
		\]
		\item \textbf{Moment bezwładności}:
		\[
		I = \sum\limits_{i = 1}^n \Delta m_i \cdot r^2_i
		\]
		\[
		I = \lim\limits_{n \rightarrow \infty} \sum\limits_{i = 1}^n \Delta m_i \cdot r_i = \int r^2 dm
		\]
		\text{Twierdzenie Steinera}
		\[
		I = I_0 + md^2
		\]
		\text{II zasada dynamiki dla ruchu obrotowego}
		\[
		\vec{M} = \sum\limits_{i = 1}^n \vec{r_i} \times \vec{F_i}
		\]
		\[
		F_i = \Delta m_i a_i = \Delta m_i \varepsilon r_i
		\]
		\[
		M = \sum\limits_{i = 1}^n r_i F_i = \sum\limits_{i = 1}^n \Delta m_i \varepsilon r_i^2 = \varepsilon \sum\limits_{i = 1}^n m_i r_i^2
		\]
		\[
		M = I \cdot \varepsilon
		\]
		\[
		\vec{M} = I \cdot \vec{\varepsilon}
		\]
		\item \textbf{Moment pędu}:
		\[
		\vec{L_i} = \vec{r_i} \times \vec{p_i} = \vec{r_i} \times \Delta m_i \cdot \vec{v_i}
		\]
		\[
		\vec{L_i} = \Delta m_i \vec{r_i} \times (\vec{\omega} \times \vec{r_i})
		\]
		\[
		\vec{\omega} = \vec{\omega_{\perp}} + \vec{\omega_{\parallel}}
		\]
		\[
		\text{Dla całej bryły: }
		L = \sum\limits_{i = 1}^n r_i \Delta m_i v_i
		\]
		\[
		v = \omega \cdot r
		\]
		\[
		L = \sum\limits_{i = 1}^n r_i \Delta m_i \omega r_i = \omega \sum\limits_{i = 1}^n \Delta m_i r_i^2 = I \omega
		\]
		\[
		\vec{L} = I \vec{\omega}
		\]
		\[
		\vec{L} = \L\vec{L_{\perp}} + \vec{L_{\parallel}} = I_{\perp} \vec{\omega_{\perp}} + I_{\parallel} \vec{\omega_{\parallel}}
		\]
		\[
		I_{\perp} = mr^2, I_{\parallel} = \frac{1}{2} mr^2
		\]
		\text{II zasada dynamiki dla ruchu obrotowego}
		\[
		\vec{M} = I \vec{\varepsilon} = I \cdot \frac{d \vec{\omega}}{dt} = \frac{d(I \cdot \vec{\omega})}{dt}
		\]
		\[
		\vec{M} = \frac{d \vec{L}}{dt}
		\]
		\text{Zasada zachowania pędu}
		\[
		\vec{M} = \frac{d \vec{L}}{dt} \text{ jeżeli } \vec{M} = \vec{0} \Rightarrow \vec{L} = \overrightarrow{const}
		\]
		\[
		L = L' \text{ tzn. } I \omega = I' \omega '
		\]
		\[
		I > I' \Rightarrow \omega ' > \omega
		\]
		\text{Elipsoida bezwładności bryły sztywnej względem jej środka masy}
		\[
		\frac{x^2}{a^2} + \frac{y^2}{b^2} + \frac{z^2}{c^2} = 1
		\]
		\[
		a = \frac{1}{\sqrt{I_x}}, b = \frac{1}{\sqrt{I_y}}, c = \frac{1}{\sqrt{I_z}}
		\]
		\item \textbf{Precesja bąka}:
		\[
		| \vec{M} | = | \vec{r} \times m \vec{g} | = rmg \sin \alpha
		\]
		\[
		r \longrightarrow \text{wektor położenia środka masy przy zetknięciu z podłożem}
		\]
		\[
		d \vec{L} = \vec{M} dt
		\]
		\[
		d \varPhi = \frac{dL}{L \sin\alpha} = \frac{Mdt}{L \sin \alpha}
		\]
		\[
		\omega_p = \frac{d \varPhi}{dt} = \frac{M}{L \sin \alpha} = \frac{mgr \sin \alpha}{L \sin \alpha} = \frac{mgr}{L}
		\]
		\[
		\vec{M} = \vec{\omega_p} \times \vec{L}
		\]
		
	\end{enumerate}
	
\end{document}
